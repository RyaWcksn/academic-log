\documentclass[12pt, a4paper, onecolumn, oneside, final, bahasa]{report}
\usepackage{graphics}
\usepackage{wrapfig}
\usepackage[indonesian]{babel}
\usepackage[T1]{fontenc}
\usepackage{tgpagella}
\usepackage{microtype}
\usepackage{booktabs}

\usepackage{upquote}
\usepackage[utf8]{inputenc}
\usepackage{eurosym}

\usepackage{upquote}
\usepackage[utf8]{inputenc}
\usepackage{eurosym}

\usepackage{amsmath}
\usepackage{amssymb}

\usepackage{graphicx}
\usepackage{float}
\let\origfigure\figure
\let\endorigfigure\endfigure
\renewenvironment{figure}[1][2] {
    \expandafter\origfigure\expandafter[H]
} {
    \endorigfigure
}

\usepackage[unicode=true]{hyperref}
\hypersetup{breaklinks=true,
            bookmarks=true,
            pdfauthor={Pramudya Arya Wicaksana},
            pdftitle={Ujian Tengah Semester},
            colorlinks=true,
            urlcolor=NavyBlue,
            linkcolor=magenta,
            pdfborder={0 0 0}}
\urlstyle{same}




\usepackage{enumerate}

\title{Ujian Tengah Semester}

\newcommand{\HRule}{\rule{\linewidth}{0.5mm}}


\providecommand{\tightlist}{%
  \setlength{\itemsep}{0pt}\setlength{\parskip}{0pt}}

 
    \usepackage{graphicx}
    % Redefine \includegraphics so that, unless explicit options are
    % given, the image width will not exceed the width or the height of the page.
    % Images get their normal width if they fit onto the page, but
    % are scaled down if they would overflow the margins.
    \makeatletter
    \def\ScaleWidthIfNeeded{%
     \ifdim\Gin@nat@width>\linewidth
        \linewidth
      \else
        \Gin@nat@width
      \fi
    }
    \def\ScaleHeightIfNeeded{%
      \ifdim\Gin@nat@height>0.9\textheight
        0.9\textheight
      \else
        \Gin@nat@width
      \fi
    }
    \makeatother

    \setkeys{Gin}{width=\ScaleWidthIfNeeded,height=\ScaleHeightIfNeeded,keepaspectratio}%


\usepackage[bf,tiny,center]{titlesec}
\titleformat{\chapter}[display]
  {\large \bfseries \centering}{\MakeUppercase{\chaptertitlename}\ \thechapter}{0pt}{\large\MakeUppercase}
\titlespacing*{\chapter}{-10pt}{-20pt}{25pt}

\begin{document}

\begin{titlepage}
    \pagenumbering{gobble}
    \begin{center}
        \begin{figure}
            \begin{center}
            % Logo
                \includegraphics[width=2.5cm]{\$HOME/notes/notes/campus/logo.png}
            \end{center}
        \end{figure}
        \vspace*{0cm}
        \textbf{STMIK AMIK Bandung} \\
        \textbf{Ujian Tengah Semester} \\[1.0cm]
        \vspace*{2.5cm}
        \textbf{UTS} \\
        \vspace*{3 cm}       
        \textbf{Pramudya Arya Wicaksana} \\
        \textbf{2242805} \\
        \vspace*{5.0cm}
        \textbf{
            FAKULTAS INFORMATIKA \\
            PROGRAM STUDI TEKNIK INFORMATIKA \\
            BANDUNG
        }
    \end{center}
\end{titlepage}
\pagenumbering{roman}
\setcounter{page}{2}


{
    \hypersetup{linkcolor=black}
    \tableofcontents
}

    \hypersetup{linkcolor=black}
    \listoffigures



\newpage
\pagenumbering{arabic}

\textbf{PENGERJAAN UJIAN LINEAR ALGEBRA UNTUK MEMENUHI KEWAJIBAN
SEMESTER 1 JURUSAN TEKNIK INFORMATIKA}\\
\textbf{PENGERJAAN MENGGUNAKAN LATEX DAN TEXT EDITOR}

\hypertarget{pertanyaan-pertama}{%
\chapter{Pertanyaan pertama}\label{pertanyaan-pertama}}

\hypertarget{soal}{%
\section{Soal}\label{soal}}

\begin{figure}
\centering
\includegraphics{/home/aya/Pictures/Screenshots/Screenshot_Wed-02_Nov_22_22.00.png}
\caption{pertama}
\end{figure}

\hypertarget{jawab}{%
\section{Jawab}\label{jawab}}

\begin{enumerate}
\def\labelenumi{\arabic{enumi}.}
\tightlist
\item
  \(\mathbf{BA}\) = \(B_{4x5} A_{4x5}\), Tidak dapat dilakukan
\item
  \(\mathbf{AB}^\intercal\) =
  \(A_{4x5} B^\intercal_{5x4} AB^\intercal{4x4}\), Terdefinisikan dengan
  ordo \(\mathbf{4x4}\)
\item
  \(\mathbf{AC+D}\) =
  \((A_{4x5}C_{5x2}) + D_{4x2} = AC_{4x2} + D_{4x2} = AC+D_{4x2}\),
  Terdefinisikan dengan ordo \(\mathbf{4x2}\)
\item
  \(\mathbf{E(AC)}\) =
  \(E_{5x4}(A_{4x5}B_{4x5}) = E_{5x4}AB_{4x5} = E(AC)_{5x2}\),
  Terdefinisikan dengan ordo \(\mathbf{5x2}\)
\item
  \(\mathbf{A-3E^\intercal}\) =
  \(A_{4x5} - 3(E^intercal_{5x4}) = A_{4x5} - 3E^\intercal_{5x4} = A-3E^\intercal_{4x5}\),
  Terdefinisikan dengan ordo \(\mathbf{4x5}\)
\end{enumerate}

\hypertarget{pertanyaan-kedua}{%
\chapter{Pertanyaan kedua}\label{pertanyaan-kedua}}

\hypertarget{soal-1}{%
\section{Soal}\label{soal-1}}

\begin{figure}
\centering
\includegraphics{/home/aya/Pictures/Screenshots/Screenshot_Wed-02_Nov_22_22.40.png}
\caption{kedua}
\end{figure}

\hypertarget{jawab-1}{%
\section{Jawab}\label{jawab-1}}

\begin{enumerate}
\def\labelenumi{\arabic{enumi}.}
\item
  \(D=\begin{bmatrix} 1 & 5 & 2 \\ -1 & 0 & 1 \\ 3 & 2 & 4 \end{bmatrix}\)
  +
  \(E=\begin{bmatrix} 6 & 1 & 3 \\ -1 & 1 & 2 \\ 4 & 1 & 3 \end{bmatrix}\)
  =
  \(\begin{bmatrix} 7 & 6 & 5 \\ -2 & 1 & 3 \\ 7 & 3 & 7 \end{bmatrix}\)
\item
  \(2\begin{bmatrix} 4 & -1 \\ 0 & 2 \end{bmatrix} -\begin{bmatrix} 1 & 4 & 2 \\ 3 & 1 & 5 \end{bmatrix}\)\\
  \(2\begin{bmatrix} 8 & -2 \\ 2 & 4 \end{bmatrix} -\begin{bmatrix} 1 & 4 & 2 \\ 3 & 1 & 5 \end{bmatrix}\)
  Matrix 2B dan C tidak bisa dikurangkan karena elementnya berbeda
  \emph{2x2} \& \emph{3x2}
\item
  \(2E=\begin{bmatrix} 6 & 1 & 3 \\ -1 & 1 & 2 \\ 4 & 1 & 3 \end{bmatrix} = E=\begin{bmatrix} 12 & 2 & 6 \\ -2 & 2 & 4 \\ 8 & 2 & 6 \end{bmatrix}\)\\
  \(-3\begin{pmatrix} \begin{bmatrix} 1 & 5 & 2 \\ -1 & 0 & 1 \\ 3 & 2 & 4 \end{bmatrix} +\begin{bmatrix} 12 & 2 & 6 \\ -2 & 2 & 4 \\ 8 & 2 & 6 \end{bmatrix} \end{pmatrix} =\begin{bmatrix} -39 & -21 & -24 \\ 9 & -6 & -15 \\ -33 & -12 & -30 \end{bmatrix}\)
\item
  \(2\begin{bmatrix} 3 & -1 & 1 \\ 0 & 2 & 1 \end{bmatrix} +\begin{bmatrix} 1 & 4 & 2 \\ 3 & 1 & 5 \end{bmatrix}\)\\
  \(\begin{bmatrix} 6 & -2 & 2 \\ 0 & 4 & 2 \end{bmatrix} +\begin{bmatrix} 1 & 4 & 2 \\ 3 & 1 & 5 \end{bmatrix}\)\\
  \(\begin{bmatrix} 7 & 2 & 4 \\ 3 & 5 & 7 \end{bmatrix}\)
\item
  \(D^\intercal-E^\intercal\)\\
  \(\begin{bmatrix} 1 & -1 & 3 \\ 5 & 0 & 2 \\ 2 & 1 & 4 \end{bmatrix} -\begin{bmatrix} 6 & -1 & 4 \\ 1 & 1 & 2 \\ 3 & 1 & 3 \end{bmatrix} =\begin{bmatrix} -5 & 0 & -1 \\ 4 & -1 & 0 \\ -1 & 2 & 1 \end{bmatrix}\)
\item
  \(\begin{pmatrix} \begin{bmatrix} 1 & 5 & 2 \\ -1 & 0 & 1 \\ 3 & 2 & 4 \end{bmatrix} \begin{bmatrix} 6 & 1 & 3 \\ -1 & 1 & 2 \\ 4 & 1 & 3 \end{bmatrix} \end{pmatrix}^\intercal =\begin{pmatrix} \begin{bmatrix} -5 & 4 & -1 \\ 0 & -1 & -1 \\ -1 & 1 & 1 \end{bmatrix} \end{pmatrix}^\intercal\)\\
  \(\begin{bmatrix} -5 & 0 & -1 \\ 4 & -1 & 1 \\ -1 & -1 & 1 \end{bmatrix}\)
\item
  \(\begin{bmatrix} 4 & -1 \\ 0 & 2 \end{bmatrix} \begin{bmatrix} 3 & 0 \\ -1 & 2 \\ 1 & 1 \end{bmatrix}\)
\end{enumerate}

Element ada matrices A dan B berbeda, jadi tidak bisa dikalikan

\begin{enumerate}
\def\labelenumi{\arabic{enumi}.}
\setcounter{enumi}{7}
\item
  \(\begin{bmatrix} 3 & 0 \\ -1 & 2 \\ 1 & 1 \end{bmatrix} \begin{pmatrix} \begin{bmatrix} 4 & -1 \\ 0 & 2 \end{bmatrix} \begin{bmatrix} 1 & 4 & 2 \\ 3 & 1 & 5 \end{bmatrix} \end{pmatrix}\)\\
  \(\begin{bmatrix} 1 & 15 & 3 \\ 6 & 2 & 10 \end{bmatrix} =\begin{bmatrix} 3 & 45 & 9 \\ 11 & -11 & 19 \\ 7 & 17 & 13 \end{bmatrix}\)
\item
  \(\begin{bmatrix} 1 & 4 & 2 \\ 3 & 1 & 5 \end{bmatrix} \begin{bmatrix} 3 & 0 \\ -1 & 2 \\ 1 & 1 \end{bmatrix} =\begin{bmatrix} 1 & 7 \\ 13 & 7 \end{bmatrix}\)
\item
  \(\begin{pmatrix} \begin{bmatrix} 1 & 5 & 2 \\ -1 & 0 & 1 \\ 3 & 2 & 4 \\ \end{bmatrix} \begin{bmatrix} 3 & 0 \\ -1 & 2 \\ 1 & 1 \end{bmatrix} \end{pmatrix}^\intercal\)\\
  \(\begin{bmatrix} 0 & 12\\ -2 & 1 \\ 11 & 8 \end{bmatrix} \begin{bmatrix} 0 & -2 & 11\\ 12 & 1 & 8 \end{bmatrix}\)
\item
  \(\begin{pmatrix} \begin{bmatrix} 3 & 0 \\ -1 & 2 \\ 1 & 1 \end{bmatrix} \begin{bmatrix} 4 & -1 \\ 0 & 2 \end{bmatrix} \end{pmatrix} \begin{bmatrix} 1 & 4 & 2 \\ 3 & 1 & 5 \end{bmatrix}\)\\
  \(\begin{pmatrix} \begin{bmatrix} 12 & -3 \\ -4 & 5 \\ 4 & 1 \\ \end{bmatrix} \end{pmatrix} \begin{bmatrix} 1 & 4 & 2 \\ 3 & 1 & 5 \end{bmatrix}\)\\
  \(\begin{bmatrix} 3 & 11 & 7 \\ 45 & -11 & 17 \\ 9 & 17 & 13 \end{bmatrix}\)
\end{enumerate}

\hypertarget{pertanyaan-ketiga}{%
\chapter{Pertanyaan ketiga}\label{pertanyaan-ketiga}}

\hypertarget{soal-2}{%
\section{Soal}\label{soal-2}}

\begin{figure}
\centering
\includegraphics{/home/aya/Pictures/Screenshots/Screenshot_Wed-02_Nov_22_23.51.png}
\caption{kedua}
\end{figure}

\hypertarget{jawab-2}{%
\section{Jawab}\label{jawab-2}}

\begin{enumerate}
\def\labelenumi{\arabic{enumi}.}
\item
  \(5x1+6x2-7x3 = 2\)\\
  \(-x1-2x2+3x3 = 0\)\\
  \(4x2-x3 = 3\)\\
  \(5x1+6x2-7x3=2\)
\item
  \(5x-3y-6z = -9\)\\
  \(2x+3y = 2\)\\
  \(3x-6z = -11\)
\item
  \(2x+3y = 2\)\\
  \(2x = 2 - 3y\)\\
  \(x = 1 - \frac{3}{2y}\)
\item
  \(x+y+z = 2\)\\
  \(1-\frac{3}{2y}+y+z = 2\)\\
  \(1-\frac{1}{2y}+z = 2\)\\
  \(\frac{1}{2y}+z = 1\)
\item
  \(5x-3y-6z = -9\)\\
  \(5(1-\frac{3}{2y})-3y-6z = -9\)\\
  \(5(\frac{15}{2y})-3y-6z = -9\)\\
  \(\frac{21}{2y}-6z = -14\)
\end{enumerate}
\end{document}
